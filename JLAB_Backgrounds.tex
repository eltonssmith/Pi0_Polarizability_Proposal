
%<><><><><><><><><><><><><><><><><><><><><><><><><><><><><><><><><><><><><><><><><><><><><>
 % Backgrounds
 %<><><><><><><><><><><><><><><><><><><><><><><><><><><><><><><><><><><><><><><><><><><><><>
\subsection{Backgrounds}

\subsubsection{Coherent and incoherent backgrounds in the reaction}

Coherent $\rho^0$ photo-production is not a background for this
experiment because $\rho^0$ decay into the $\pi^0\pi^0$ channel is
prohibited by I-spin conservation.  The largest coherent background is
from $f_0(500)$ and $f_0(980)$ photo-production.  The width of the
$f_0(980)$ is from 10 to 100~MeV, and can be eliminated from the data
by a cut on $\pi^0\pi^0$ invariant mass.  The $f_0(500)$ width is much
broader, from 400 to 700 MeV, with significant overlap in the
invariant mass region of interest.  Since the $f_0(500)$ is a scalar
particle with the same spin-parity as the $\gamma \gamma \rightarrow
\pi^0\pi^0$ final state near threshold, the azimuthal distribution of
the $\pi^0$ momentum or the $\pi^0\pi^0$ c.m. momentum relative to the
photon polarization plane does not differentiate between coherent
$f_0(500)$ production and the Primakoff reaction.  This is similar to
the Primex-$\pi^0$ experiment, where the dominant background was
nuclear coherent $\pi^0$ photo-production.  The approach used in the
Primex analysis was to measure the $\pi^0$ angular distribution,
effectively the $t$-distribution, then use theoretical calculations of
the angular distributions to separate out contributions from Primakoff
and nuclear coherent. The analysis of the $\pi^0\pi^0$ (NPP) reaction
will approximately parallel what was done for the Primex-$\pi^0$
analysis.

Primex data also showed that the nuclear coherent process is highly
suppressed for heavy nuclei.  The reason for the suppression is
$\pi^0$ absorption in the nuclear interior, making the coherent
production primarily a surface effect, i.e. proportional to $A$ and
not $A^2$.  For NPP it is expected that suppression of the nuclear
coherent will be stronger than that seen in Primex because two pions
are produced in NPP as compared to a single $\pi^0$ in Primex.  NPP
plans to run on a heavy nuclear target such as $^{208}$Pb.

The inelastic and incoherent reactions that might contribute to the
data include
\begin{enumerate}[label=(\roman*)]
    \item nuclear coherent production of $\eta$ followed by $\eta\rightarrow \pi^0\pi^0\pi^0 \rightarrow \gamma\gamma\gamma\gamma(\gamma\gamma)$, where two of the six decay photons go unobserved
    \item $\gamma N \rightarrow N \pi^0\pi^0$
\end{enumerate}

The first reaction is an inelastic, coherent process, and as such
could produce a significant rate for a heavy nuclear target. Rejecting
events with extra gammas in the final state would suppress this
background.  The second reaction is an incoherent process, and is
small relative to coherent processes.  The Primex analysis showed that
incoherent reactions generally peak at large angles relative to the
Primakoff peak, and had a small effect on the extraction of the
Primakoff $\pi^0$ cross sections.