\section{Parameterization of the nuclear coherent production  \label{sec:NCsigma}}
We consider the reaction $\gamma A \rightarrow m_{\pi\pi} A$, where $m_{\pi\pi}\rightarrow \pi\pi$ is a dipion system.
The $2\pi$ system is treated as a particle with mass $m_{\pi\pi}$, which is produced with  four-momentum transfer $t$.  
The cross section for a three-body final state can be written as \cite{BNLQGS020900}:
\begin{eqnarray}
d\sigma & = & \frac{1}{4\mathcal{F}} \, d\phi_3 \, |\mathcal{A}|^2 \\
\mathcal{F} & = & p_\gamma^{cm} \sqrt{s} \\
d\phi_3 & = & \frac{4}{(4\pi)^5} {p_\sigma^{cm} \over \sqrt{s}} d\Omega_\sigma^{cm}\, p_\pi^{\sigma} dm_{\pi\pi} d\Omega_\pi^{\sigma} \\
{dt \over d\Omega_\sigma^{cm} } & = & {dt \over d\cos{\theta}_\sigma^{cm} d\phi_\sigma^{cm} } = {2\, p_\gamma^{cm} p_\sigma^{cm}  \over d\phi_\sigma^{cm} }
\end{eqnarray}
The center-of-mass energy (cm) energy and the momentum transfer are represented by the commonly used variables $s$ and $t$. 
%The invariant mass of the 2$\pi$ system ($\sigma$) is denoted by W$_{\pi\pi}$. 
Other variables are subscripted by particle name and their superscripts indicate the reference frame. Thus $p_\gamma^{cm}$ is the incident photon momentum, $p_\sigma^{cm}$ is the scattered momentum,  and $\Omega_\sigma^{cm}$ corresponds to the solid angle of the $\sigma$, all in the cm frame.  
The momentum of the pions in the $\sigma$ rest frame is denoted by $p_\pi^{\sigma}$ and $\Omega_\pi^{\sigma}$ denotes the solid angle of  one of them.
Thus the cross section can be written as
\begin{eqnarray}
{d\sigma \over dt dm_{\pi\pi} d\phi_{\sigma}^{cm} d\Omega_\pi^{\sigma} } & = & {1 \over 2(4\pi)^5} {p_\pi^\sigma \over (p_\gamma^{cm})^2 s } |\sum_i \mathcal{A}^i|^2 ,     \label{eq:sigma_phi3}
\end{eqnarray}
where the index $i$ runs over the number of resonances or mechanisms included in the calculation. We  will assume that we can parameterize each production amplitude as a factorized product
\begin{eqnarray}
\mathcal{A}^i & = & \mathcal{A}_t(t)^i \, \mathcal{A}_W(m_{\pi\pi})^i \, \mathcal{A}_\tau(\Phi, \phi, \theta)^i.
\end{eqnarray}
For simplicity, we will drop the superscript $i$ since for the moment we are considering  single production mechanism.
The function $\mathcal{A}_\tau(\phi_{\pi\pi}, \phi_\pi, \theta_\pi)$ contains the angular dependence of the produced pions, 
where\,($\theta_\pi,\phi_\pi$)  are the decay angles in the rest frame of the $2\pi$ system, which is flat for S-wave production. 
Azimuthal symmetry is broken by the photon polarization, where $\phi_{\pi\pi}$\,is the angle between the plane of photon polarization and the production plane. The amplitudes are
given by Eq.\,\ref{eq:PhiAmplitude} and lead to a cross section dependence of the form $\mathcal{A}_\tau \propto (1 + \mathcal{P} \cos{2\phi_{\pi\pi}}$). 

The primary background in this mass region is given by the 
f$_0(500)(J^{PC}=0^{++})$  also called the $\sigma$. The $\sigma$ has the same angular structure as the Primakoff 
reaction and can only be identified through its dependence on $t$ and $m_{\pi\pi}$.
Our parameterization of the mass dependence for the $\sigma$ meson is described in Section\,\ref{sec:ParmSwave}.

We assume the $-t$ dependence of the $\sigma$ has a similar form as for single $\pi^0$ production, namely $\mathcal{A}_t(t) \propto \sin{\theta_{\pi\pi}} \times F_{st}(t)$.  The $\sin{\theta_{\pi\pi}}$ 
comes from the spin-flip required at forward angles to produce a $0^+$ system from a spin-zero target. The factor $F_{st}(t)$ is the strong form factor for the target, which is approximated to
match calculations for the single $\pi^0$ production (Fig.\,6 from Ref.\cite{Gevorkyan:2009ge}). 



 
\subsection{Parameterization of the s-wave amplitude \label{sec:ParmSwave}}
There is considerable strength in the $2\pi$ channel coming from s-wave production, which is due to the now established $f_0(500)$ meson. It is also commonly referred to as the $\sigma$ 
meson. We assume the amplitude for $\sigma$ production  
is governed by the $\pi\pi$ J=0, I=0 phase shifts. We parameterize the $m_{\pi\pi}$ dependence as 
\begin{eqnarray}       
\mathcal{A}_W(m_{\pi\pi}) & \sim & \frac{m_{\pi\pi}}{2k} \sin{\delta_0} e^{i\delta_0} \left(\alpha_1 + \alpha_2 m_{\pi\pi}^2 \right) + 
 \cos{\delta_0} e^{i\delta_0} \left(\alpha_3 + \alpha_4 m_{\pi\pi}^2 \right),                    \label{eq:Aw}
\end{eqnarray}
where $\delta_0$ is the s-wave phase shift for $I=0$ and $\alpha_i$ (i=1, 2, 3, 4) are empirical constants to be obtained from data.
The first term is due to ``compact source'' production of the pion pair 
(see Eq.\,5 from Ref.\,\cite{Pelaez:2015qba}) and the second term is due to 
production due to an ``extended source,'' for example pion rescattering (see Eq.\,5 from Ref.\,\cite{Bibrzycki:2018pgu} and Eq.\,9 from Ref.\,\cite{Aitchison:1977sm}).  
We use the parameterization for the s-wave phase shifts 
from Appendix D of Ref.\,\cite{Ananthanarayan:2000ht}:\footnote{See also Eq.\,44 of Ref.\,\cite{Pelaez:2015qba}.}  
\begin{eqnarray}  
\tan{\delta_0} & = &  \frac{2k}{m_{\pi\pi}} \left(A_{0}^0 + B_{0}^0 k^2 + C_{0}^0 k^4 + D_{0}^0 k^6 \right) \left(4m_\pi^2 - s_{0}^0 \over M_{\pi\pi}^2 - s_{0}^0 \right), 
\end{eqnarray}
where we use the same notation as the reference with $A_{0}^0$ =0.225, $B_{0}^0$ =12.651\,GeV$^{-2}$, $C_{0}^0$= -43.8454\,GeV$^{-4}$,   $D_{0}^0$=-87.1632\,GeV$^{-6}$, and $s_0^0$=0.715311\,GeV$^2$.
We have converted the constants to units of GeV and evaluated the parameters for $a_0^0=0.225\,m_\pi^{-1}$, and $a_0^2=-0.0371\,m_\pi^{-1}$. These fits are only valid below $m_{\pi\pi}<$ 0.9 GeV because
they do not properly include the $f_0(980)$. 

The empirical constants in Eq.\,\ref{eq:Aw} were determined by fitting $|\mathcal{A}_W|^2$ to the S-wave contribution to the photoproduction 
cross section\footnote{The data are available through the Durham HEP Databases, http://durpdg.dur.ac.uk/.}  
measured by CLAS for $E_\gamma=3-3.8$ GeV \cite{Battaglieri:2009aa} for $-t=0.4-0.5$ GeV$^2$. The fits are for $m_{\pi\pi}=0.3-0.95$ GeV, which is our region of interest.
All four parameters are needed to obtain a good representation to the central values of the data, although the uncertainty band in the data allow for a wide range of parameters. 
Assuming that the constants are real and relatively independent of energy and $-t$,
we take the average of the fitted constants for our parameterization ($\alpha_1=8.4\pm1.4$,  $\alpha_2=-4.1\pm2.2$,  $\alpha_3=2\pm1.1$,  $\alpha_4=8\pm1.1$).

\section{Angular distribution in the helicity basis}

\subsection{Photon density matrix in the helicity basis}
The linear polarization of the photon can be expressed as (Ref.\cite{Schilling:1969um}  Eq. 18-19):
\begin{eqnarray}
\rho(\gamma) & = & \textstyle{1 \over 2} I + \textstyle{1 \over 2} \vec{P_{\gamma}} \cdot \vec{\sigma}, \, \rm{where} \label{eq:photon_sdmH} \\
\vec{P_\gamma} & = & \mathcal{P} (-\cos 2\phi_{\pi\pi}, -\sin 2\phi_{\pi\pi}, 0)
\end{eqnarray}
and $\vec{\sigma}$ are the Pauli matrices. The angle $\phi_{\pi\pi}$ is the angle between the polarization vector of the photon and the production plane and $\mathcal{P}$
represents the degree of linear polarization. Multiplying out these factors gives the expression for the 
photon density matrix in the helicity frame as (Ref.\cite{Salgado:2013dja}  Eq. 219):
\begin{eqnarray}
\rho_{\epsilon,\epsilon'} (\gamma) & = & \textstyle{\frac{1}{2}} \left( \begin{array}{cc} 1 & -\mathcal{P}e^{-2i\phi_{\pi\pi}} \\
-\mathcal{P}e^{2i\phi_{\pi\pi}} & 1 \end{array} \right)    \label{eq:dm_photon}
\end{eqnarray}

\subsection{Parity constraints  \label{sec:parity}}
We consider the reaction $a + b \rightarrow c + d$, where the spin of each particle is denoted by $s_j$, their helicity by $\lambda_j$ and their intrinsic parity by $\eta_j$. 
If parity is conserved, there are relations between amplitudes with opposite helicities, which are  given in Jacob and Wick \cite{Jacob:1959at} Eq. 43 and  Ref.\cite{Schilling:1969um}  Eq. 20 \footnote{We thank Adam
Szczepaniak for clarifying the connection between these papers.} (see also Ref. \cite{leader} Eq. 4.2.3):
\begin{eqnarray}
^{\lambda_a}V_{\lambda_c}^{\lambda_{d}\lambda_{b}}  & = & \left( \eta_c \eta_d \over \eta_a \eta_b \right) (-1)^{s_c+s_d-s_a-s_b} (-1)^{(\lambda_c-\lambda_d)-(\lambda_a-\lambda_b)}  \hspace{0.25cm} ^{-\lambda_a}V_{-\lambda_c}^{-\lambda_{d}-\lambda_{b}}   \label{eq:parity}
\end{eqnarray}

\subsection{S-wave production}
For the case of S-wave production of two pions via the $f_0(500)$ or $\sigma$ meson off an spinless target we have the following constraint:
\begin{eqnarray}
^{\lambda_\gamma}V_{\lambda_\sigma}^{\lambda_{Z}\lambda_{Z}} & = & ^{\epsilon}V_{0}^{00} = \left( \eta_c+ \over -+ \right) (-1)^1 (-1)^{-1} \,\,^{-\epsilon} V_0^{00} = - \eta_c\,^{-\epsilon}V_0^{00}
\end{eqnarray}
For convenience, we have separated out the parity of the scattered state $\eta_c$. 
The $2\pi$ intensity distribution (see Ref.\cite{Salgado:2013dja} Eq. 220-223 and also Eqs. 264) is given by the following expression after 
dropping the superscripts  related to the target helicities and collapsing 
the sums over external and internal spins  because both the target and resonance are $0^+$ objects:
\begin{eqnarray}
\mathcal{I} & = & \sum\limits_{\epsilon \epsilon'} \, ^\epsilon V_0 \, Y_0^0 \, \rho_{\epsilon \epsilon'} \, ^{\epsilon'}V_0^* \, Y_0^{0*}   \label{eq:swave_intensity}\\
 & = & \textstyle{\frac{1}{2}}  |Y_0^0|^2     \left( ^1V_0 \hspace{0.3cm} ^{-1}V_0 \right) \, \left( \begin{array}{cc} 1 & -\mathcal{P}e^{-2i\phi_{\pi\pi}} \\
-\mathcal{P}e^{2i\phi_{\pi\pi}} & 1 \end{array} \right)  \left( \begin{array}{c} ^1V_0^* \\ ^{-1}V_0^* \end{array} \right) \\
& = &  \textstyle{\frac{1}{2}}  |Y_0^0|^2  \left[ |^1V_0|^2 - \mathcal{P} \,^{1}V_0 \,^{-1}V_0^* \, e^{-2i\phi_{\pi\pi}} - \mathcal{P} \,^{1}V_0^* \,^{-1}V_0\, e^{2i\phi_{\pi\pi}}  +  |^{-1}V_0|^2  \right]
\end{eqnarray}
Noting that $^{1}V_0 = -\eta_c\,^{-1}V_0$, we obtain the following expression:
\begin{eqnarray}
\mathcal{I} & = &  \textstyle{\frac{1}{4\pi}}   |^1V_0|^2 \left(1 + \eta_c \mathcal{P} \cos{2\phi_{\pi\pi}} \right),
\end{eqnarray} 
where $\phi_{\pi\pi}$ is the angle of the polarization vector relative to the production plane. For the case of $\sigma$ production, $\eta_c = +1$, but for the case of $\pi^0$ production we have
the opposite sign, $\eta_c = -1$.
%\footnote{This analysis gives the correct sign for $\pi^0$ production, but does not include the $\Sigma$ asymmetry as a factor; not sure why. The formulas for Primakoff production do not include such a factor.}    
For the Primakoff production of $\pi^+\pi^-$ in S-wave, $\eta_c = (-1)(-1)(-1)^0 = +1$. See Ref. \cite{CPPexp} Eq. 8. 

The intensity distribution in Eq.\,\ref{eq:swave_intensity} may be written in a more convenient form for use with AmpTools, namely
\begin{eqnarray}
\mathcal{I} & = & (\textstyle{1-\mathcal{P} \over 4}) |A_{+}|^2 +  (\textstyle{1+\mathcal{P} \over 4}) |A_{-}|^2 \\
A_{\pm} & = & Y^0_0 (^{1}V_0 \pm \,^{-1}V_0 \, e^{2i\phi_{\pi\pi}} )\\
A_{\pm} & = & Y^0_0 \, ^{1}V_0\,(1 \mp \eta_c\, e^{2i\phi_{\pi\pi}}),
\end{eqnarray}
which can be written more symmetrically taking advantage of an arbitrary phase as
\begin{eqnarray}
A_{\pm} & = & Y^0_0 \, ^{1}V_0\,(e^{-i\phi_{\pi\pi}}  \mp \eta_c\, e^{i\phi_{\pi\pi}}).  \label{eq:PhiAmplitude}
\end{eqnarray}